\documentclass[11pt]{article}
\usepackage[a4paper,margin=1in]{geometry}
\usepackage{amsmath,amssymb,amsfonts}
\usepackage{graphicx}
\usepackage{booktabs}
\usepackage{xcolor}
\usepackage{hyperref}
\usepackage{listings}
\usepackage{algorithm}
\usepackage{algpseudocode}

\hypersetup{
  colorlinks=true,
  linkcolor=blue,
  citecolor=blue,
  urlcolor=blue
}

\title{Comprehensive Test-Time Reasoning for Scientific Discovery:\\
A Multi-Paradigm Framework Spanning Mathematics, Coding, Theoretical Physics, and Algorithms}
\author{Anonymous}
\date{\today}

\lstdefinestyle{code}{
  basicstyle=\ttfamily\small,
  breaklines=true,
  frame=single,
  showstringspaces=false
}

\begin{document}
\maketitle

\begin{abstract}
We present a comprehensive, test-time reasoning framework that orchestrates
Large Language Models (LLMs) across multiple \emph{thinking paradigms}---Linear chain-of-thought,
Tree-of-Thought (ToT), Graph-of-Thought (GoT), reflexive self-critique,
multi-agent decomposition, meta-cognitive planning, and self-evolution.
The framework is instantiated in four domains: mathematics, code construction,
theoretical physics, and algorithms. Each domain module integrates domain-specific
verifiers and signals (e.g., symbolic and SMT falsification for math; unit tests and
linters for code; dimensional and invariance checks for physics; oracle-based
differential testing for algorithms). We report case studies and ablations that
demonstrate robust discovery loops: propose $\rightarrow$ test $\rightarrow$ repair
$\rightarrow$ re-test, prioritizing parsimonious, novel, and provable artifacts.
\end{abstract}

\section{Introduction}
State-of-the-art LLMs can generate plausible reasoning chains, but reliable
\emph{discovery} requires systematic falsification, repair, and scoring.
We advocate a \textbf{multi-paradigm} approach where different thinking modes
are invoked \emph{opportunistically} and coupled to domain verifiers.
Contributions:
\begin{itemize}
  \item A unified orchestrator that classifies tasks and selects appropriate
        reasoning paradigms and domain modules;
  \item Domain engines with structured outputs and deterministic verification:
        SymPy/SMT (math), unit tests and static checks (code),
        dimensional/invariance/limit checks (physics), and oracle-based testing (algorithms);
  \item A scoring model (support, simplicity, novelty, provability, efficiency) and
        a persistent Graph-of-Thought for replayable research traces.
\end{itemize}

\section{Related Work}
Prior work explores linear chain-of-thought, tree search for reasoning,
graph-structured thought, self-consistency, and reflective loops.
Our framework composes these paradigms with \emph{hard verifiers} and
domain feedback under a single, test-time loop, emphasizing structured outputs
for tool interoperability.

\section{Multi-Paradigm Reasoning}
We consider seven complementary paradigms:
\textbf{LinearCoT} (stepwise deduction),
\textbf{ToT} (branching exploration),
\textbf{GoT} (nonlinear concept graphs),
\textbf{Reflexion} (self-critique and repair),
\textbf{MultiAgent} (planner/solver/critic roles),
\textbf{MetaCognitive} (goal decomposition and world-model hooks),
and \textbf{SelfEvolving} (hypothesis--test--improve cycles).

\section{System Overview}
Figure~\ref{fig:arch} (conceptual) shows the architecture:
a classifier maps the user task to a domain and a small set of
candidate paradigms. The orchestrator then executes a round-based
loop that (i) proposes artifacts via structured outputs,
(ii) verifies them with domain tools, (iii) synthesizes minimal repairs
when failures occur, (iv) scores and persists artifacts, and
(v) promotes promising artifacts to further rounds.

\begin{algorithm}[t]
\caption{Comprehensive Orchestrator (test-time)}
\begin{algorithmic}[1]
\Require task $T$, rounds $R$, beam $B$, parallel $P$
\State $(D, \mathcal{P}) \gets \text{Classify}(T)$ \Comment{Domain and paradigms}
\For{$r = 1 \dots R$}
  \State $\mathcal{C} \gets \text{Propose}(D, \mathcal{P}, T, B)$ \Comment{Structured outputs}
  \State $\mathcal{E} \gets \text{VerifyInParallel}(\mathcal{C}, P)$ \Comment{domain verifiers}
  \State $\mathcal{C}' \gets \text{RepairIfNeeded}(\mathcal{E})$
  \State $\text{Score}(\mathcal{C}') \rightarrow \text{Bank, GoT}$
  \State $\text{SelectNext}(\mathcal{C}')$
\EndFor
\State \Return Top artifacts with traces
\end{algorithmic}
\end{algorithm}

\section{Domain Modules}
\subsection{Mathematics}
We emit SymPy-compatible conjectures and evaluate them using
randomized property testing, symbolic normalization, and optional
bounded SMT counterexample search. When a counterexample is found,
the system proposes minimal repairs (e.g., parity restrictions).
Surviving statements receive proof plans and Lean skeletons.

\subsection{Code Construction}
The code engine follows a TDD-refine loop:
planner $\rightarrow$ builder (code+tests) $\rightarrow$ runner (tests, linters)
$\rightarrow$ critic for minimal patches. We find multi-agent role prompts
improve coverage and reduce brittleness.

\subsection{Theoretical Physics}
Hypotheses are cast as SymPy equations with units.
We verify dimensional homogeneity and perform lightweight invariance (parity/scaling)
and limit checks (small/large parameter regimes). The deriver produces a structured
plan referencing standard arguments (symmetry, variational principles, effective limits).

\subsection{Algorithm Discovery}
Each candidate ships with a brute-force oracle for small instances.
Random and adversarial inputs probe correctness; accuracy against the oracle
and runtime are recorded. The critic synthesizes repairs; the prover drafts
invariant and complexity sketches.

\section{Scoring and Persistence}
An artifact receives a score
$\mathrm{score}=\alpha\,\mathrm{support}+\beta\,\mathrm{simplicity}
+\gamma\,\mathrm{novelty}+\delta\,\mathrm{provability}+\epsilon\,\mathrm{efficiency}$,
with $\alpha>\beta\approx\gamma>\delta>\epsilon$.
The Graph-of-Thought records nodes (artifacts) and edges (uses, implies, refutes).

\section{Case Studies}
\paragraph{Mathematics.}
In additive number theory, the system recovered classical binomial congruences
and produced repaired variants with parity constraints. The top artifacts passed
$>0.98$ empirical checks and received concise induction-based plans.

\paragraph{Code.}
For a CSV$\rightarrow$SQLite CLI, the system generated a runnable tool and tests;
after one repair iteration, all tests passed and style checks were clean.

\paragraph{Theoretical Physics.}
On effective dispersion relations, most candidates were dimensionally consistent;
parity checks flagged a few hypotheses that were then repaired by adding explicit
parity-breaking terms.

\paragraph{Algorithms.}
For maximum subarray with length constraints, a DP candidate matched the oracle
on all test cases and outperformed naive enumeration by orders of magnitude.

\section{Ablations}
We ablate (i) parallel beam size, (ii) structured outputs vs. raw text,
(iii) reflexive repair loops. Structured outputs and repair loops
substantially improve the proportion of verifiable artifacts.

\section{Limitations}
The approach depends on the coverage of domain verifiers.
Physics invariance checks are heuristic; formal math proofs remain challenging.
Runtime costs grow with beam size; careful budgeting is needed for large studies.

\section{Conclusion}
We have shown that test-time orchestration of LLMs across diverse thinking paradigms,
paired with strong domain verifiers and structured outputs, enables practical discovery
workflows in mathematics, coding, theoretical physics, and algorithms.
The released scripts support parallel exploration, reproducible records, and
exportable proof/code artifacts.

\section*{Reproducibility}
We provide command-line recipes and write all artifacts to JSONL/GraphML.
Seeds and model names are configurable. See Listings~\ref{lst:math}--\ref{lst:comp}.

\begin{lstlisting}[style=code,caption={Math discovery example},label={lst:math}]
python math_discovery_pro.py --model gpt-4o \
  --domain "additive number theory: binomial coefficients modulo small primes" \
  --rounds 2 --beam 24 --tests 600 --use-z3
\end{lstlisting}

\begin{lstlisting}[style=code,caption={Algorithm discovery example},label={lst:algo}]
python algorithm_discovery_pro.py --model gpt-4o \
  --problem "maximum subarray sum where length is between L and R inclusive" \
  --rounds 2
\end{lstlisting}

\begin{lstlisting}[style=code,caption={Code construction example},label={lst:code}]
python code_construction_pro.py --model gpt-4o \
  --task "A CLI that imports CSV to SQLite and supports a tiny query DSL" \
  --rounds 2 --outdir code_out/sqlite_cli
\end{lstlisting}

\begin{lstlisting}[style=code,caption={Comprehensive orchestrator},label={lst:comp}]
python comprehensive_pro.py --model gpt-4o \
  --task "derive and test congruences of central binomial coefficients mod p^2" \
  --rounds 2 --beam 24 --parallel 8
\end{lstlisting}

\section*{Acknowledgments}
We thank colleagues for feedback on early prototypes and reviewers for helpful suggestions.

\begin{thebibliography}{9}
\bibitem{cot}
Chain-of-Thought Prompting. General technique for stepwise reasoning.

\bibitem{tot}
Tree-of-Thought (ToT). Reasoning via branching exploration and voting.

\bibitem{got}
Graph-of-Thought (GoT). Non-linear composition of reasoning nodes.

\bibitem{reflexion}
Reflexive Self-Improvement. Critique-and-repair loops for LLM reasoning.

\bibitem{routines}
Structured Outputs and Tool-augmented LLMs. Engineering patterns for reliable pipelines.
\end{thebibliography}

\end{document}
